\label{sec:design}
\begin{center}
\begin{table*}[t!]
   \centering
   \begin{tabular}{|l|l|l|l|}
      \hline
      \bf Battery & \bf Capacity (mAh) & \bf Weight, Manufacturer (g) & \bf Energy Density (mAh/g) \\
      \hline
      3xAAA NiMH & 800 & 36 & 22 \\
      \hline
      Sparkfun 850mAh & 850 & 18.5 & 46 \\
      \hline
      Sparkfun 400mAh & 400 & 9 & 44 \\
      \hline
      Sparkfun 110mAh & 110 & 2.65 & 41.5 \\
      \hline
      Tenergy 3.7V 500mAh & 500 & 12 & 42 \\
      \hline
      Tenergy 3.7V 1000mAh & 1000 & 13 & 76 \\
      \hline
      Trustfire 10440 & 600 & 9.6 & 63 \\
      \hline
      Trustfire 10440 (enclosed) & 600 & FIXME & FIXME \\
      \hline
   \end{tabular}
   \caption[b]{Rechargeable Battery Roundup.  This table shows a roundup of some off-the-shelf rechargeable batteries which were considered for our glasses prototype.  All batteries in the chart are lithium except the NiMH AAAs.  Their weights, capacities and an "energy density" attribute (capacity per weight, measured in mAh/g), are shown.  The energy densities have been computed with the measured weight for each battery except the enclosed Trustfire 10440.  All battery setups listed run at nominal voltages of 3.6-3.7V, and all lithium batteries listed contain protection circuitry.}
   \label{tab:batteries}
\end{table*}
\end{center}
\section{System Design}
In this section, we present the design of our wireless eyeglasses.  Many important constraints must be considered including weight, battery life (power draw), data aquisition rate and the on glasses processing capabilities. 
\subsection{Weight}
% FIXME: what is the typical weight we found in a survey of our collegues eyeglasses?  What was the maximum.
When developing a wearable computing device, one of the first constraints to consider is weight.  Our eyeglasses must be worn on the user's head like traditional eyeglasses, and therefore weight is an important constraint.  In a brief survey of collegues' spectacles, we found weights ranging from 15-50g.  In order for a pair of eyeglasses to be feasible for all day use, they must weight something close to this range.

In addition to a maximum weight constraint, a pair of glasses has a balance constraint.  Accomplishing a balanced pair of glasses is as easy as adding weight to the lighter side until an acceptable balance is reached.  However, since we want to minimize weight, the glasses should be designed to be close to balanced.  Our primary sources of weight are the circuit board and battery.  We simply decided to place the battery on one leg and the circuit board on the other.

The battery for the glasses is one of the heavier components, so we start our estimate of system weight by comparing several several candidate batteries.  With our choice of battery, we have a trade-off between conflicting constraints, weight and energy capacity (which feeds directly into battery life).  We capture this interaction by dividing capacity (in mAh) per unit weight (in g), resulting in an "energy density" with units mAh.g.  A comparison of potential batteries and capacities is provided in table ~\ref{tab:batteries}.  Each COTS battery considered contains a protection circuit and runs at a nominal 3.6-3.7V.

% FIXME ref: hermera board.
% FIXME weigh the sunglasses
Finally, we want a baseline weight for our circuit board.  We have taken the Hermera board as an example of a similarly sized and weighted circuit board.  The hermera board weighs Xg.  This is heavier than some of the batteries in table ~\ref{tab:batteries} and lighter than others.  Again, we will assume the system weight to be roughly double the larger of the two weights, plys the weight of the sunglasses chassis.  Our sunglasses weight Xg.

\subsection{Battery Life / Power}

%In order to show that the day of ubiquitous camera devices is truly here, we must consider the power bounds on such a system.b
%
%Our second primary constraint in the weight / battery-life tradeoff is, of course, battery-life.  

% FIXME this isn't even readable.

We want to take a look at the power constraints on a pair of wireless glasses.  Wireless glasses have unique constraints.  These batteries are just a selection of off the shelf parts which are reasonable capacities, voltages and weights for an embedded device running on a pair of glasses.  It gives an idea of what kind of power draw our system really needs to achieve the goal of all day use.

Again, table ~\ref{tab:batteries} shows examples of COTS lithium rechargeable batteries.  All of the batteries have protection built in and run at a nominal voltage of 3.6-3.7V.  The heaviest (but not highest capacity) battery is 

Because we intend our device to be worn all day potentially, we can come up with an initial baseline for battery life.  We will assume that all day entails 20 hrs in the worst case.  So, to get our energy needs, we merely need an estimate of our system's power draw.  If, for example, our system draws 50mA on average, the glasses would require at least a 1000mAh battery to achieve 20hrs of battery life, assuming perfect efficiency.

\subsubsection{Cameras}

% FIXME: references for the OmniVision OV7670 and the aptina
For a long time, cameras have been power hungry.  In recent years we've seen proliferation of lower and lower power cameras, and they are now nearly ubiquitous within cell phones.  Now, we are finally starting to see truely low power cameras enter the market.  Typical CMOS low power cameras draw in the several 10s of mW range.  Fr example the Aptina MT9V11 ultra low-power camera draws 80mW in active mode, and the Omnivision OV7670 draws 60mW, in both cases at 15 FPS and VGA resolution.

% FIXME is this actually true?
Another factor of energy cost in most CMOS cameras is the required complexity necessary when interfacing with them.  Often, the camera runs at a set FPS and data must be captured according tight timing constraings.  This requires more complexity from the embedded interfacing device, and this almost always translates to more power.

We are focusing low-power image sensing on truly low-power.  In this usecase (a go-everywhere wearable computing device), we need a camera which can do significantly better thanthese VGA options, but we're willing to sacrifice image quality to get that.  An exciting new camera, the CentEye Stonyman, is capable of running at any speed (completely asyncronous).  We have taken some measurements of the CentEye Stonyman camera and reveal it's very modest energy consumption when active, capturing an image.

% FIXME: TODO: put in the stonyman measurements.  Do some calculations on them.


% TODO: remove this old stuff:
%\subsection{Computational Eyeglasses}

%\subsection{Processing Device}

